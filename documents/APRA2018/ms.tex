\documentclass[12pt]{article}

\usepackage{microtype}
\usepackage{natbib}
\usepackage{hyperref}
\usepackage{graphicx}
\usepackage{booktabs}
\usepackage{xspace}
\usepackage{paralist}
\usepackage{amsmath}
\usepackage{subcaption}
\usepackage[dvipsnames]{xcolor}
\usepackage[final]{pdfpages}

\usepackage{ragged2e}
\usepackage[format=plain,justification=RaggedRight,font=footnotesize]{caption}

\usepackage[style=plain,floatrowsep=qquad]{floatrow}

\setlength{\parskip}{\baselineskip}
\setlength{\parindent}{0em}

\usepackage{sectsty}
\allsectionsfont{\raggedright\bfseries\large}
\subsectionfont{\raggedright\normalfont\itshape\normalsize}

\usepackage[compact]{titlesec}

\usepackage{aas_macros}
\usepackage{amsmath}

\usepackage[charter]{mathdesign}

\usepackage[margin=1in]{geometry}

\usepackage{fancyhdr}
\pagestyle{fancyplain}
\renewcommand{\headrulewidth}{0pt}
\lhead{}
\chead{}
\rhead{}
\lfoot{}
\cfoot{\thepage}
\rfoot{}

\hypersetup{pdfpagemode={UseOutlines},
  bookmarksopen,
  colorlinks,
  linkcolor={Black},
  citecolor={Black},
  urlcolor={RoyalBlue}}

\bibliographystyle{apj}

\newcommand*{\eg}{{e.\,g.,}\xspace}
\newcommand*{\ie}{{i.\,e.,}\xspace}

\newcommand{\todo}[1]{{\color{red} TODO: #1}}



%%%%%%%%%%%%%%%%%%%%%%%%%%%%%%%%%%%%%%%%%%%%%%%%%%%%%%%%%%%%%%%%%%%%%%%%%%%%%%%%
%% fix a bug in bibtex: only the year in author-year should be a link
%%
\RequirePackage{etoolbox}
\makeatletter

% Patch case where name and year have no delimiter
\patchcmd{\NAT@citex}
  {\@citea\NAT@hyper@{\NAT@nmfmt{\NAT@nm}\NAT@date}}
  {\@citea\NAT@nmfmt{\NAT@nm}\NAT@hyper@{\NAT@date}}
  {}% Do nothing if patch works
  {}% Do nothing if patch fails

% Patch case where name and year have basic delimiter
\patchcmd{\NAT@citex}
  {\@citea\NAT@hyper@{%
     \NAT@nmfmt{\NAT@nm}%
     \hyper@natlinkbreak{\NAT@aysep\NAT@spacechar}{\@citeb\@extra@b@citeb}%
     \NAT@date}}
  {\@citea\NAT@nmfmt{\NAT@nm}%
   \NAT@aysep\NAT@spacechar%
   \NAT@hyper@{\NAT@date}}
  {}% Do nothing if patch works
  {}% Do nothing if patch fails

% Patch case where name and year are separated by a prenote
\patchcmd{\NAT@citex}
  {\@citea\NAT@hyper@{%
     \NAT@nmfmt{\NAT@nm}%
     \hyper@natlinkbreak{\NAT@spacechar\NAT@@open\if*#1*\else#1\NAT@spacechar\fi}%
       {\@citeb\@extra@b@citeb}%
     \NAT@date}}
  {\@citea\NAT@nmfmt{\NAT@nm}%
   \NAT@spacechar\NAT@@open\if*#1*\else#1\NAT@spacechar\fi%
   \NAT@hyper@{\NAT@date}}
  {}% Do nothing if patch works
  {}% Do nothing if patch fails

\makeatother
%% end bugfix
%%%%%%%%%%%%%%%%%%%%%%%%%%%%%%%%%%%%%%%%%%%%%%%%%%%%%%%%%%%%%%%%%%%%%%%%%%%%%%%%

% Alter some LaTeX defaults for better treatment of figures:
% See p.105 of "TeX Unbound" for suggested values.
% See pp. 199-200 of Lamport's "LaTeX" book for details.
% General parameters, for ALL pages:
\renewcommand{\topfraction}{0.9}	% max fraction of floats at top
\renewcommand{\bottomfraction}{0.8}	% max fraction of floats at bottom
%
% Parameters for TEXT pages (not float pages):
\setcounter{topnumber}{2}
\setcounter{bottomnumber}{2}
\setcounter{totalnumber}{4}             % 2 may work better
\setcounter{dbltopnumber}{2}            % for 2-column pages
\renewcommand{\dbltopfraction}{0.9}	% fit big float above 2-col. text
\renewcommand{\textfraction}{0.07}	% allow minimal text w. figs
%
% Parameters for FLOAT pages (not text pages):
\renewcommand{\floatpagefraction}{0.7}	% require fuller float pages
%
% N.B.: floatpagefraction MUST be less than topfraction !!
\renewcommand{\dblfloatpagefraction}{0.7} % require fuller float pages


\renewcommand*{\maketitle}{
\begin{center}
  \begingroup
  \Large
  \textbf{Technology development for cubesat-based near-infrared exoplanet transit spectroscopy} \\*[0.5\baselineskip]
  \normalsize
\begin{tabular}{r@{:\quad}lr}
          PI & Michael Gully-Santiago       & Bay Area Environmental Research Institute       \\
        Co-I & Joeseph Lin & MIT Lincoln Laboratory
\end{tabular}
  \endgroup
\end{center}
}


\begin{document}
\raggedbottom
\fontsize{12}{16}\selectfont

\pagenumbering{roman}

%%%%%%%%%%%%%%%%%%%%%%%%%%%%%%%%%%%%%%%%%%%%%%%%%%%%%%%%%%%%%%%%%%%%%%%%%%%%%%%%
%% TABLE OF CONTENTS

\maketitle

\tableofcontents


%%%%%%%%%%%%%%%%%%%%%%%%%%%%%%%%%%%%%%%%%%%%%%%%%%%%%%%%%%%%%%%%%%%%%%%%%%%%%%%%
%% SUMMARY

\clearpage

\addcontentsline{toc}{section}{Summary}

\maketitle

2.3.3 Proposal Summary (abstract)

We propose development of a near-infrared spectrograph concept devoted to exoplanet transit spectroscopy of bright exoplanet host stars.  The compact design of this spectrograph will fit inside of a low-cost cubesat platform, dramatically lowering the overall instrument cost. The device development for this concept must occur now in order to be ready for discoveries from present and future transiting exoplanet discovery missions (K2/TESS).

We will develop two key enabling technologies of our instrument concept.  First, we will develop custom cross-dispersed silicon immersed diffractive grisms to achieve both high spectral grasp, and a small overall instrument volume amenable to a cubesat platform.  Our team has considerable expertise in developing silicon immersion grating technology.  Second, we will develop intra-pixel sensitivity characterization of near-IR detectors.  Intra-pixel sensitivity characterization provides resilience to spacecraft-induced thermal instabilities, which lead to the dominant noise sources in both Kepler/K2 and likely for TESS as well.  The dearth of lab-measured sub-pixel detector characterization currently limits many science cases in K2 data, for example.  Through the APRA program, we will develop lab-based metrology techniques and concepts for in-situ, space-based calibration systems.  Through these device development and metrology efforts, we will enable hundreds of parts per million transit spectroscopy on the sample of bright, TESS-discovered planets.



%%%%%%%%%%%%%%%%%%%%%%%%%%%%%%%%%%%%%%%%%%%%%%%%%%%%%%%%%%%%%%%%%%%%%%%%%%%%%%%%
%% Scientific/Technical/Management Section

\cleardoublepage
\setcounter{page}{1}
\pagenumbering{arabic}

\addcontentsline{toc}{section}{Scientific/Technical/Management}
\addtocontents{toc}{\protect\setcounter{tocdepth}{-1}}

%% title
\maketitle

2.3.6 Scientific/Technical/Management Section

\section{Objectives and expected significance of the proposed research}

\subsection{Science case}
- cite exoplanet atmospheres and decadal proposal
- cite JWST science cases
- importance of distinguishing planets by their atmospheric composition

(in order for bibtex to work, i have to cite something\ldots here you
go: \citealt{Einstein1936})


\section{Methodology}
\label{sec:method}
%%
\subsection{Proposal Prompt}

\begin{quote}
The technical approach and methodology to be employed in conducting
the proposed research, including a description of any hardware
proposed to be built in order to carry out the research, as well as
any special facilities of the proposing organization(s) and/or
capabilities of the proposer(s) that would be used for carrying out
the work. (Note: ref. also Section 2.3.10(a) concerning the
description of critical existing equipment needed for carrying out the
proposed research and the Instructions for the Budget Justification in
Section 2.3.10 for further discussion of costing details needed for
proposals involving significant hardware, software, and/or ground
systems development, and, as may be allowed by an FA, proposals for
flight instruments);
\end{quote}

\subsection{Sensitivity demands for exoplanet transit spectroscopy}
The most favorable transiting exoplanet GJ1214b produces an average transit depth of 1.4\%.  \citet{2015ApJ...815..110M} predicted the transit depth as a function of wavelength expected for a GJ1214b-like planetary transit.  Figure \ref{fig:Morley} shows an overview of the transit spectrum and a zoom on the $J-band$ region at the native resolution of the synthetic spectral models.  The transit spectrum shown in Figure \ref{fig:Morley} assumes 100$\times$ the stellar metallicity, with no clouds in the atmosphere.  The native standard deviation of the $J-$band spectrum is 700 parts per million.


\begin{figure}
    \centering
    \begin{subfigure}[b]{0.45\textwidth}
        \includegraphics[width=\textwidth]{../../figures/GJ1214b_wl_depth}
        \caption{Panchromatic model}
        \label{fig:GJ1214b_panchromatic}
    \end{subfigure}
    ~ %add desired spacing between images, e. g. ~, \quad, \qquad, \hfill etc.
      %(or a blank line to force the subfigure onto a new line)
    \begin{subfigure}[b]{0.45\textwidth}
        \includegraphics[width=\textwidth]{../../figures/GJ1214b_wl_depth_zoom}
        \caption{$J-band$ region}
        \label{fig:GJ1214b_zoom}
    \end{subfigure}
    \caption{The GJ1214b transit depth as a function of wavelength}\label{fig:Morley}
\end{figure}

\textbf{What are the brightest known transiting-planet host stars? What will TESS find?}

The top 20 candidates for exoplanet transit spectroscopy have recently been compiled in Table 3 of \cite{2017AJ....153..256R}, an abridged repication of which appears here for ease of clarity. We expect signal strengths in the ~100 parts per million range for targets in this list.

\begin{table}
 \centering
 \caption{The Best Confirmed Planets for Transmission Spectroscopy with R$_P$ $<$ 5$R_{\oplus}$, and $J<8$, with GJ1214b for reference.  }
 \label{tbl:StoN}
 \begin{tabular}{cccc}
    \hline
    \hline
    Planet & R$_P$($R_{\oplus}$) & S/N$^a$ &$J$ (mag) \\
    \hline
 GJ 1214 b  & 2.85$\pm$0.20 & 1.00 & 9.75\\
 GJ 436 b  & 4.1697408 & 0.68 & 6.9\\
 %GJ 3470 b  & 3.88$\pm$0.33 & 0.48 & 8.8 \\
 HAT-P-11 b  & 4.73$\pm$0.26 & 0.46 &7.6 \\
 55 Cnc e  & 1.91$\pm$0.08 & 0.41 & 4.59\\
 HD 97658 b  & 2.34$^{+0.17}_{-0.15}$ & 0.30 & 6.2\\
 HD 3167 c & 2.85$^{+0.24}_{-0.15}$ & 0.26 & 7.5\\
 %HD 106315 c & $4.3_{-0.3}^{+0.2}$ & 0.22 & 8.1\\
 %K2-25 b  & 3.43$^{+0.95}_{-0.31}$ & 0.22 & 11.3\\
 HIP 41378 d  & 3.96$\pm$0.59 & 0.1 &7.9\\
 HIP 41378 b  & 2.90$\pm$0.44 & 0.1 &7.9\\
 %K2-32 d  & 3.76$\pm$0.40 & 0.13 & 10.4\\
 %K2-19 c  & 4.86$^{+0.62}_{-0.44}$ & 0.12 &11.5 \\
 %K2-28 b  & 2.32$\pm$0.24 & 0.1 & 11.7\\
 %K2-32 c  & 3.48$^{+0.98}_{-0.42}$ & 0.12 & 10.4\\
 %Kepler-105 b  & 4.81$\pm$1.5 & 0.11 & 11.8\\
 %Kepler-411 c  & 3.27$^{+0.12}_{-0.067}$ & 0.11 & 10.6\\
 %HD 106315 b & 2.5$\pm$0.1 & 0.10 & 8.6\\
   \hline
    \hline
 \end{tabular}
\begin{flushleft}
 \footnotesize{ \textbf{\textsc{NOTES:}}
$^a$The predicted signal-to-noise ratios relative to GJ 1214 b.
}
\end{flushleft}
\end{table}

\textbf{How small of a telescope can you measure such an exoplanet transit with?}

A $D=\,7.5$ cm telescope, with spectral resolution $R=1000$, total efficiency $\epsilon=0.1$ and 30 electrons read noise would achieve a mere 3.6 $S/N$.  GJ1214b is out of reach.  However, much brighter targets remain accessible.  Table \ref{tbl:StoN} lists the best known transiting planet systems amenable to transit spectroscopy with a small telescope, benchmarked to GJ1214b.  TESS will discover many more such systems.  Figure \ref{fig:StoNj} shows the expected $S/N$ for a baseline $J-band$ spectrograph design.

\begin{figure}
    \centering
        \includegraphics[width=0.5\textwidth]{../../figures/SN_vs_J}
    \caption{The Expected signal-to-noise ratio for a $J-$band spectrograph}\label{fig:StoNj}
\end{figure}


\textbf{Why is transit spectroscopy so hard?}
- The integrated solid angle of the planet atmosphere is generally much less than the star's disk, yielding low signal-to-noise ratio of the atmospheric signal.
- Many consecutive observations are required to build up enough signal to noise ratio
- Systematic changes in the spectrograph can confound the disentangling of minuscule instrumental artifacts and planet-transit signals
- Unmodelled astrophysical variations (starspots / plages) also confound retrieval of planetary atmosphere signals. (cite Apai recent paper on Trappist 1).

\subsection{Intrapixel sensitivity variations}
16. What are intrapixel sensitivity variations, and how does they affect precision measurements?

Although much emphasis has focused on detector *read noise*, less metrology effort has focused on intrapixel sensitivity variations (cite ongoing work at RIT / Dmitri).  These minute perturbations are the main confounding factor in the unbiased analysis of Kepler/K2 data, and will limit TESS similarly.  Other factors like detector rolling band, and cross-talk also hamper our unbiased analysis of Kepler data.  By characterizing these instrumental signals, we will be able to forward model the generating process of our data acquisition.  We will have excellent generative models of both instrumental systematics, and excellent data-driven models of the stellar atmosphere from many repeated observations of transit-free spectra.  The final ingredient, and the desired science target---transiting exoplanet atmospheric spectra---will be inferred with a powerful probabilistic graphical model capable of detecting weak, correlated signals from messy residuals (cite Czekala et al. 2015, 2017, Gully-Santiago et al. 2017.)

\textbf{What is the advantage of going to space?}
- Stability
- Continuous monitoring (point and stare)
- Low/no background or telluric absorption

\textbf{Other}
- Demonstrate the low-cost cubesat platform.
- Demonstrate resilient strategies for calibrating instruments in Earth orbit.
- Innovations in software and modeling.
- Simulate a transit spectrum retrieval
- See Caroline Morley simulations


\section{Cubesat concept}


\textbf{What is the nominal design of the spectrograph?}
- Largest possible aperture that can fit in the cubesat
- Si Grism
- Custom MIT Lincoln Lab detector

\textbf{What are possible orbital configurations?}
- Either Earth orbiting (TESS) or Earth trailing (Kepler, CUTE)
- Earth orbiting preferred for communications

\textbf{What is the nominal operation of a cubesat with the proposed design?}
- Identify a few or even just one good bright target(s)
- Adapt orbit to continuously view targets continuously or for ~month-long campaigns
- Just point and stare and collect tens of thousands of spectra
- Model the in- and out- of spectra transits

\textbf{What are the best targets?}
- Would need to be bright, as already shown
- Would need to have planets with close-in orbits to increase the number of transits viewable in a campaign duration
- Such planets would be highly irradiated.


\subsection{Our APRA 2018 proposal}
What will you actually do for APRA 2018?  Why are technical barriers important?
- We will develop to two key enabling technologies for a compact near-IR spectrograph cubesat concept.
- Bonded cross-dispersed silicon immersed grisms for high efficiency, high resolution, high bandwidth, compact spectrograph design.
- Extreme precision detector metrology to calibrate detector systematics, enabling us to foward model the data acquisition process.
- Silicon immersed gratings deliver higher spectral grasp for a given beam size than conventional optical materials.  The >2x size savings reduces cost for a given desired spectral resolution, making a compact, low-cost cubesat design possible.  However, bonded, silicon grisms have not yet been demonstrated in space (their single)


\textbf{Who will do the work?  Where?}

- MGS will develop the Si grism technology at either UTexas, MIT Lincoln Lab, NASA Ames, or JPL.
- JL will develop the extreme precision detector metrology at MIT Lincoln Lab
- MGS will supervise Ames personnel will build simulations of correlated instrumental noise impact on exoplanet atmospheric retrievals.


\section{Impact}

\textbf{What is the unique, new science}
- Measuring the near-IR spectrum of a planet (still few examples)
- Detect water from molecular bands in the near-IR

\subsection{Comparison to existing and planned capabilities}

How does this mission compare with existing or proposed missions?
- JWST  (slated for launch)
- CUTE (in development)
- FINESSE (proposed)
- Ground-based (Recent Giano paper, others)

\subsection{Unique qualifications of the team}
Why is your team uniquely suited to carrying out the proposed research program?

- MGS led development of Si grisms and immersion gratings for his PhD.  He developed a Fabry-Perot inspired metrology technique for detecting gaps as small as 10 nm in bonded Si optics (cite Gully-Santiago; APRA NNX)

- JL has pioneered custom ASICs and advanced detector applications for over a decade.


\section{Management Plan}
\label{sec:management}
%%
A general plan of work, including anticipated key milestones for
accomplishments, the management structure for the proposal personnel,
any substantial collaboration(s) and/or use of consultant(s) that
is(are) proposed to complete the investigation; and a description of
the expected contribution to the proposed effort by the PI and each
person as identified in one of the additional categories in Section
1.4.2, regardless of whether or not they derive support from the
proposed budget.




%%%%%%%%%%%%%%%%%%%%%%%%%%%%%%%%%%%%%%%%%%%%%%%%%%%%%%%%%%%%%%%%%%%%%%%%%%%%%%%%
%% DATA MANAGEMENT PLAN

\cleardoublepage

\addtocontents{toc}{\protect\setcounter{tocdepth}{1}}

\section*{Data Management Plan}
\addcontentsline{toc}{section}{Data Management Plan}
%%


Our team practices open source software and data sharing best practices.  Most of our data analysis software development occurs in the open on the code sharing website GitHub.  The permissive MIT licenses allow all code and data to be reused freely for any application, including commercial.  In some cases, code and materials are developed in private for quality assurance, licensing, or export control purposes.  In these rare cases, we continue to adopt distributed version control systems (\emph{i.e.} \texttt{git}), and plan curated open sources releases at a later date when content quality, licensing, and export control can be addressed.

We carry out research and development data management best practices with conventional backups on external disks, RAID, and cloud services where appropriate and allowed by the nature of the content.  Scientific data of broader appeal will be made available publicly, and could be made available via popular long-lifetime storage solutions like the Mikulski Archive for Space Telescopes (MAST) if the scientific value was deemed worthy.  Commercial storage options like Amazon Web Services (AWS), Google Cloud Platform (GCP), and others could be pursued for their ease of low-cost computation, and seamless interface.



%%%%%%%%%%%%%%%%%%%%%%%%%%%%%%%%%%%%%%%%%%%%%%%%%%%%%%%%%%%%%%%%%%%%%%%%%%%%%%%%
%% REFERENCES AND CITATIONS

\cleardoublepage

\addcontentsline{toc}{section}{References and Citations}

\bibliography{ms}


%%%%%%%%%%%%%%%%%%%%%%%%%%%%%%%%%%%%%%%%%%%%%%%%%%%%%%%%%%%%%%%%%%%%%%%%%%%%%%%%
%% BIBLIOGRAPHICAL SKETCHES

\cleardoublepage

\section*{Biographical Sketches}
\addcontentsline{toc}{section}{Biographical Sketches}
%
2.3.8 Biographical Sketch(s) [Ref.: Appendix B: Part (c)(6)]

The PI (and Co-PI) must include a biographical sketch (not to exceed
two pages) that includes his/her professional experiences and
positions and a bibliography of recent publications, especially those
relevant to the proposed investigation. A one-page sketch for each Co-
Investigator must also be included (Note: Any Co-I also serving in one
of the three special Co-I categories defined in Section 1.4.2 may use
the same two-page limit as for the PI). For the PI and any Co-Is who
are required to provide Current and Pending Support information
(ref. Section 2.3.8), the biographical sketch must include a
description of scientific, technical and management performance on
relevant prior research efforts. Those participants who will play
critical management or technical roles in the proposed investigation
should demonstrate that their qualifications, capabilities, and
experience are appropriate to provide confidence that the proposed
objectives will be achieved.

\cleardoublepage
\includepdf[pages={1,2}]{cvs/cv.pdf}


%%%%%%%%%%%%%%%%%%%%%%%%%%%%%%%%%%%%%%%%%%%%%%%%%%%%%%%%%%%%%%%%%%%%%%%%%%%%%%%%
%% CURRENT AND PENDING SUPPORT

\cleardoublepage

\section*{Current and Pending Support}
\addcontentsline{toc}{section}{Current and Pending Support}
%%
2.3.9 Current and Pending Support [Ref.: Appendix B, Part (c)(10)]

Information must be provided for all ongoing and pending projects and
proposals that involve the proposing PI. This information is also
required for any Co-Is who are proposed to perform a significant share
(>10 percent) of the proposed work.

All current project support from whatever source (e.g., Federal,
State, local or foreign government agencies, public or private
foundations, industrial or other commercial organizations) must be
listed. This information must also be provided for all pending
proposals already submitted or submitted concurrently. Do not include
the current proposal on the list of pending proposals unless it also
has been submitted elsewhere.

All projects or activities requiring a portion of the investigators'
time during the period of the proposed effort must be included, even
if they receive no salary support from the project(s). For the entire
period of the proposed award the total amount received by that
investigator (including indirect costs) or the amount per year if
uniform (e.g., \$50 K/year) must be shown as well as the number of
person-months per year to be devoted to the project for each year,
regardless of source of support.

Specifically, for the PI and any Co-Is who are proposed to perform a
significant share (>10\%) of the person's time, provide the following
information:
\begin{itemize}
\item Title of award or project title;
\item Name of PI on award;
\item Program name (if appropriate) and sponsoring agency or organization,
  including a point of contact with his/her telephone number and email
  address;
\item Performance period;
\item Total amount received by that investigator (including indirect
  costs) or the amount per year if uniform (e.g., \$50 k/year); and
\item Commitment by PI or Co-I in terms of person-months per year for each
  year.
\end{itemize}

For pending research proposals involving substantially the same kind
of research as that being proposed to NASA in this proposal, the
proposing PI must notify the NASA Program Officer identified for the
FA immediately of any successful proposals that are awarded any time
after the proposal due date and until the time that NASA's selections
are announced.  Current and pending support is not required for
student or foreign Co-Is.


%%%%%%%%%%%%%%%%%%%%%%%%%%%%%%%%%%%%%%%%%%%%%%%%%%%%%%%%%%%%%%%%%%%%%%%%%%%%%%%%
%% STATEMENTS OF COMMITMENT AND LETTERS OF SUPPORT

\cleardoublepage

\section*{Statements of Commitment and Letters of Support}
\addcontentsline{toc}{section}{Statements of Commitment and Letters of Support}
%%
2.3.10 Statements of Commitment and Letters of Support

Every Co-PI, Co-I, and Collaborator (ref. definitions in Section
1.4.2) identified as a participant on the proposal's cover page and/or
in the proposal's Scientific/Technical/Management Section must
acknowledge his/her intended participation in the proposed effort.

The NSPIRES proposal management system allows for participants named
on the Proposal Cover Page to acknowledge electronically a statement
of commitment. Although we prefer all team members to confirm
participation via NSPIRES, if that is not possible the inclusion of a
statement of commitment in the proposal as set out in the example
below may be permitted instead.

The Summary of Solicitation for an FA may specify that signed
statements of commitment must be included within the proposal. Also,
any proposals submitted via Grants.gov must include signed statements
of commitment in the proposal. In the case of more than one Co-PI,
Co-I or Collaborator at the same institution, a single statement
signed by all participants may be submitted. In any case, each
statement must be addressed to the PI, may be a facsimile of an
original statement or the copy of an email (the latter must have
sufficient information to unambiguously identify the sender), and is
required even if the Co-PI, Co-I or Collaborator is from the proposing
organization. An example of such a statement follows:

``I (we) acknowledge that I (we) am (are) identified by name as
Co-Principal Investigator(s), Co-Investigator(s) [and/or
Collaborator(s)] to the investigation, entitled <name of proposal>,
that is submitted by <name of Principal Investigator> to the NASA
funding announcement<alpha-numeric identifier>, and that I (we) intend
to carry out all responsibilities identified for me (us) in this
proposal. I (we) understand that the extent and justification of my
(our) participation as stated in this proposal will be considered
during peer review in determining in part the merits of this
proposal. I (we) have read the entire proposal, including the
management plan and budget, and I (we) agree that the proposal
correctly describes my (our) commitment to the proposed
investigation.'' For the purposes of conducting work for this
investigation, my participating organization is <<insert name of
organization>>.''

In addition, a letter of support is required from the owner of any
facility or resource that is not under the PI's direct control,
acknowledging that the facility or resource is available for the
proposed use during the proposed period. For Government facilities,
the availability of thefacility to users is often stated in the
facilities documentation or web page. Where the availability is not
publicly stated, or where the proposed use goes beyond the publicly
stated availability, a statement, signed by the appropriate Government
official at the facility verifying that it will be available for the
required effort, is sufficient.

Letters of support do not include ``letters of affirmation'' (i.e.,
letters that endorse the value or merit of a proposal). NASA neither
solicits nor evaluates such endorsements for proposals. The value of a
proposal is determined by peer review. If endorsements are submitted,
they may not be submitted as an appendix. They must be included as
part of the proposal and must be included within the required page
limitations even though they will not be considered in the evaluation
of the proposal.


%%%%%%%%%%%%%%%%%%%%%%%%%%%%%%%%%%%%%%%%%%%%%%%%%%%%%%%%%%%%%%%%%%%%%%%%%%%%%%%%
%% BUDGET JUSTIFICATION

\cleardoublepage

\section*{Budget Justification}
\addcontentsline{toc}{section}{Budget Justification}
%%
2.3.11 Budget Justification: Narrative and Details [Ref.: Appendix B, Part (c)(8)]

\begin{quote}
Each proposal shall provide a budget justification for each year of
the proposed effort and shall be supported by appropriate narrative
material and budget details in compliance with the following
instructions.

Failure to adequately provide detailed cost data will require NASA
procurement personnel to contact the proposing organization for the
required information. All proposers are required to submit a
thoroughly detailed cost breakdown. NASA procurement personnel must be
able to determine that all proposed costs are allowable, allocable,
and reasonable. All proposed costs must be directly related to the
approved project and scope of work. A detailed budget will facilitate
this cost analysis. Reference Grant and Cooperative Agreement Manual
(GCAM) located at the following URL:
\url{https://prod.nais.nasa.gov/pub/pub_library/srba/index.html}

2.3.11(a) Required Budget Narrative (Including Personnel and Work
Effort and Facilities and Equipment)

The Budget Narrative should clearly state the type of award instrument
the proposer anticipates receiving if selected for award (i.e.,
contract, grant or cooperative agreement). NASA will, however, make
the final decision on the award instrument used (reference D.1.2) and
that decision may result in the need for additional information
regarding requested items of costs

The Budget Narrative must describe the basis of estimate and rationale
for each proposed component of cost, including direct labor,
subcontracts/subawards, consultants, other direct costs (including
travel), and facilities and equipment. The proposer must provide
adequate budget detail to support estimates. The proposer must state
the source of cost estimates (e.g., based on quote, on previous
purchases for same or similar item(s), cost data obtained from
internet research, etc.) including the company name and/or URL and
date if known, but need not include the actual price quote or screen
captures from the web. The proposer must describe in detail the
purpose of any proposed travel in relation to the award and provide
the basis of estimate, including information or assumptions on
destination, number of travelers, number of days, conference fees, air
fare, per diem, miscellaneous expenses, etc. If destinations are not
known, the proposer should, for estimating purposes, make reasonable
assumptions about the potential destination and use historical cost
data based on previous trips taken or conferences attended. Any
conference, for which travel funding is requested, must be directly
related to the project.

A required element of the Budget Narrative is a table of Personnel and
Work Effort, summarizing the work effort required to perform the
proposed investigation. Proposals for contracts must include a table
of work effort in the budget narrative section, before the section on
facility costs. Proposals for grants or cooperative agreements are not
required to submit the table of work effort in the budget narrative
section, see section 2.3.12 for more details. The table must have the
names and/or titles of all personnel necessary to perform the proposed
effort, regardless of whether those individuals require funding. For
each individual, list the planned work commitment to be funded by
NASA, per period in fractions of a work year. In addition, include
planned work commitment not funded by NASA, if applicable. Where names
are not known, include the position, such as postdoc or technician.

The final element of the Budget Narrative is a description of any
required facilities and equipment. This section should describe any
existing facilities and equipment that are required for the proposed
investigation. It must explain the need for items costing more than
\$5,000 and describe the basis for estimated cost (i.e., competitive
quotes were obtained, justification for sole source purchase, proposed
cost based on previous purchases for same or similar item(s), cost
data obtained from internet research, etc.).

Proposed costs for purchased facilities, tooling, or equipment must be
entered in the Proposal Cover Page and included in the Budget Details
(ref. Section 2.3.10(b)). Proposals submitted via Grants.gov should
include a single Facilities and Equipment section as a separate PDF
document; it should be uploaded to the Grants.gov application as the
``Facilities and Other Resources'' document. ``Equipment'' document should
not be uploaded to Grants.gov.

There should be direct and obvious correlation between the items
described in the Budget Narrative, those given in the Budget Details,
and the figures entered in the Proposal Cover Page/Grants.gov forms.

2.3.11 (b) Required Budget Details

In addition to the Budget Narrative, proposers are required to include
detailed budgets, including detailed subcontract/subaward budgets, in
a format of their own choosing. Regardless of format chosen, the
following information must be included in the Budget Details.

1. Direct Labor (salaries, wages, and fringe benefits): List the
   number and titles of personnel, amounts of time to be devoted to
   the grant (level of effort for each position), and rates of
   pay. The annual salary should be clearly noted for each
   position. Labor should be clearly broken out from fringe
   benefits. The fringe benefit rate/percent should be clearly noted
   on the budget for each labor category for ease of review.

Important Note: All Recipients are reminded that in accordance with 2
CFR \S 200.414, NASA is required to apply the applicable negotiated
rate for all grants awarded to the recipient. If fringe benefits
comprise part of the applicable negotiated rate, NASA will use this
rate for all grants and cooperative agreements awarded to the
recipient. . If the prosing organization does not have negotiated rate
for fringe benefits, recipients should use their rates for fringe
benefits that is applied to funds from all funding sources.



2. Other
Direct Costs:

a. Subcontracts/Subawards: Attachments shall describe the work to be
subcontracted/subawarded, estimated amount, recipient (if known), and
the reason for subcontracting (e.g., uniquely qualified
co-investigator is located at another institution from the proposing
institution). Itemized budgets are required for all
subcontracts/subawards, regardless of dollar value.

b. Consultants: Identify consultants to be used and provide the amount
of time they will spend on the project and rates of pay to include
annual salary, overhead, etc.

c. Equipment: List all facilities and equipment items
separately. General-purpose equipment (i.e., personal computers and/or
commercial software) valued at or above \$5,000 is not allowable as a
direct cost unless specifically approved by the NASA Award
Officer. Any requested general-purpose equipment purchase valued at or
above \$5,000 to be made as a direct charge under this award must
include the equipment description, an explanation of how it will be
used in the conduct of the research proposed, and a written
certification that the equipment will be used exclusively for the
proposed research activities and not for general business or
administrative purposes. [Ref.: Appendix B, Part (c)(7)].

d. Supplies: Provide general categories of needed supplies, the method
of acquisition, and the estimated cost.

e. Travel: Provide a detailed breakout of costs for any proposed
travel. Detailed budget data shall include the following:

- Number of people and number of days - Departure/Arrival cities -
  Airfare - Per diem

- Car rental - Conference fees (if applicable) - Miscellaneous Costs
  (i.e., car rental fuel, airport parking, tolls, etc.).

Note: Every effort should be made to accurately estimate and detail
travel costs. Under Federal procurement regulations, missing or
minimum data is not acceptable for budget evaluation and award
purposes. If destinations are not known at time of proposal
preparation, then reasonable assumptions about the potential
destination and historical data for previous trips may be used but the
preparer is still required to include the same amount of detail listed
above. That is, use reasonable assumptions and historical data for
destinations and length of stay, however, use current pricing for the
applicable categories listed above. If adequate budget detail is not
submitted with the proposal then this will delay your award.

f. Other: List and enter the total of direct costs not covered by 2a
through 2e.



3. Facilities and Administrative (F\&A)/Indirect Costs: Identify F\&A
   cost rate(s) and base(s) as approved by the cognizant Federal
   agency, including the effective period of the rate. Provide the
   name, address, and telephone number of the Federal agency official
   having cognizance. If approved audited rates are not available,
   provide the computational basis for the indirect expense pool and
   the corresponding allocation base for each proposed rate. For
   grants and cooperative agreements(only): Any non-Federal entity
   that has never received a negotiated indirect cost rate, except for
   these non-Federal entities described in Appendix VII to Part
   200-paragraph D.1.b, may elect to charge a de minimis rate of 10%
   of modified total direct costs (MTDC) which may be used
   indefinitely.

Reference Important Note in paragraph 2.3.10(b)1. above: All budgets
shall be prepared using the most current ``approved'' indirect rates for
estimating and award purposes. proposers shall not use unapproved
``future'' rates. Failure to do so will cause a delay in receiving your
award as the NASA Procurement Office will then have to come back to
the proposer with a request to reduce the proposed rates to the most
current ``approved'' rates. Proposers may charge less than the approved
current rates but shall not propose more in anticipation of the rates
changing in the future. However, should the negotiated rate change
throughout the period of performance that grantee is required to apply
that changed rate to any direct funds expended during that time frame
stated on the changed agreement.


4. Other Applicable Costs: Enter total explaining the need for each
   item and itemized lists detailing expenses within major budget
   categories. Also enter here the required funding for any Co-Is who
   cannot be funded through the PI award (e.g. because the PI is at a
   non- Government organization and a Co-I is at a U.S. Government
   organization) (see Section 2.3.10(c)(ii)(a)).

5. Subtotal-Estimated Costs: Enter the sum of items 1 through 4.

Less: Proposed Cost Sharing (if any): Neither NSPIRES nor Grants.gov
allows for notating cost sharing on the standardized budget
form. However, if cost sharing is proposed, it should be discussed in
detail in the Budget Narrative. Further, if cost sharing is based on
specific cost items, identify each item and amount in the Budget
Detail with a full explanation provided in the Budget Narrative.

If an institution of higher education, hospital, other non-profit
organization wants to receive a grant or cooperative agreement, cost
sharing may not be required (review FA to see if cost is
required). The award would be made in accordance with the requirements
of 2 CFR 1800 and the Grants and Cooperative Agreement Manual. These
two documents also applicable to NASA grants and cooperative
agreements awarded to commercial firms which do not involve cost
sharing. This does not prohibit voluntary cost sharing. NASA may
accept cost sharing from any type of organization if it is voluntarily
offered. Reference 2 CFR \S200.306 (Cost Sharing or matching). If a
commercial organization wants to receive a grant or cooperative
agreement, cost sharing is required unless the commercial organization
can demonstrate that it does not expect to receive substantial
compensating benefits for performance of the work. If this
demonstration is made, cost sharing is not required but may be offered
voluntarily. Reference also 2 CFR \S1800.922 and 14 CFR \S1274.204,
(Costs and Payments), paragraph (b), Cost Sharing.

Cost sharing is not required when a commercial organization receives a
contract, but it may be offered voluntarily.


6. Total Estimated Costs: Enter the total amount of funding requested
   from the Government. 2.3.11(c) Other Budget Guidelines

In preparing the Budget Justification (both Narrative and Details),
proposers must consider the following additional important NASA
procurement policies:

(i) Purchase of Personal Computers and/or Software. Note the
discussion of item "2.c. Equipment" on the Instructions above
regarding the proposed purchase of personal computers and/or
commercial software at or above \$5,000. Such items are usually
considered by NASA to be general purpose equipment that must be
purchased from general, organizational overhead(Indirect or F\&A)
budgets and not directly from the proposal budget unless it can be
demonstrated that such items are to be used uniquely and only for the
proposed research. If a proposal is selected for award, failure to
adequately address the requirements of the instructions for item 2.c
above (Equipment) will require that NASA contact the proposing
organization for the required information. Such activity may delay the
award until the purchase is justified as a direct charge for
general-purpose equipment to be used exclusively for the proposed
research activities.

(ii) Joint Proposals Involving a Mix of U.S. Government and
Non-Government Organizations.

(a) Unless otherwise specified in the FA, if a PI from any type of
private or public organization proposes to team with a Co-I from
and/or use a facility at a U.S. Government organization (including
NASA Centers and the Jet Propulsion Laboratory), the budget for the
proposal must include all funding requested from NASA for the proposed
investigation, and this must be reflected in the budget totals that
appear in the budget forms (e.g., Proposal Cover Page, Grants.gov
forms, Budget Details). Any required budget for that Government Co-I
and/or facility should be included in the proposal's Budget Narrative
and should be listed as "Other Applicable Costs" in the required
Budget Details. If the proposal is selected, NASA will execute an
inter- or intra-Agency transfer of funds, as appropriate, to cover the
applicable costs at that Government organization.

(b) If a PI from a U.S. Government organization (including NASA
Centers and the Jet Propulsion Laboratory) proposes to team with a
Co-I from a non-Government organization, then the proposing Government
organization must cover those Co-I costs through an appropriate award
for which that Government PI organization is responsible. Such
non-Government Co-I costs should be entered as a
"Subcontract/Subaward" on the Budget Summary.

(c) If a PI from a non-U.S. organization proposes to team with a Co-I
from a U.S. organization then reference part (vii) below.

(iii) Responsibility of the Proposing Organization to Place Subawards
for Co-Is at Other Organizations. Other than the special cases
discussed in item (ii) above, and unless specifically noted otherwise
in the FA, the proposing PI organization must subcontract the funding
of all proposed Co-Is who reside at other non-Government
organizations, even though this may result in a higher proposal cost
because of subcontracting fees.

(iv) Full-Cost Accounting at NASA Centers. Regardless of whether
functioning as a team lead or as a team member, personnel from NASA
Centers must propose budgets based on full-cost accounting. Proposal
budgets from NASA Centers must include all costs that will be paid out
of the resulting award. Costs which will not be paid out of the
resulting award, but are paid from a separate NASA budget (e.g.,
Center Management and Operations, CM\&O) and are not based on the
success of this specific award, should not be included in the proposal
budget. For example, CM\&O should not be included in the proposal
budget while direct civil service labor, travel, service pools, and
other charges to the proposed research task should be
included. Proposal budgets having JPL participation should include all
costs except the JPL fixed-fee (formerly JPL award fee).

(v) Unallowable Costs. Subpart E, Cost Principles, 2 CFR \S200.400, et
seq., and the Federal Acquisition Regulation (FAR) at 48 CFR Part 31
\url{https://acquisition.gov/far/current/html/FARTOCP31.html}), identify and
describe certain costs that may not be included in a proposed budget
(unallowable costs). The use of appropriated funds for such purposes
is unallowable and may lead to cancellation of the award and possible
criminal charges. Grant recipients should be aware of cost principles
applicable to their organization as set forth in the above
regulations.

Payment of fee or profit is consistent with an activity whose
principal purpose is the acquisition of goods and services for the
direct benefit or use of the United States Government, rather than an
activity whose principal purpose is assistance. Therefore, the Grants
Officer shall use a procurement contract, rather than assistance
instrument, in all cases where fee or profit is to be paid to the
recipient of the instrument or the instrument is to be used to carry
out a program where fee or profit is necessary to achieving program
objectives. Grants and Cooperative Agreements shall not provide for
the payment of fee or profit to the recipient.

(vi) Prohibition of the Use of NASA Funds for
Non-U.S. Research. NASA's policy welcomes the opportunity to conduct
research with non-U.S. organizations on a cooperative, no-
exchange-of-funds basis. Although Co-Is or collaborators employed by
non-U.S. organizations may be identified as part of a proposal
submitted by a U.S. organization, NASA funding may not typically be
used to support research efforts by non-U.S. organizations at any
level, including travel by foreign investigators. However, the direct
purchase of supplies and/or services that do not constitute research
from non-U.S. sources by U.S. award recipients is
permitted. Ref. Section (l) of Appendix B. Also reference paragraph
(c)(8)(iv) of Appendix B which states in part, ``NASA funding may not
be used for foreign research efforts at any level, whether as a
collaborator or a subcontract. The direct purchase of supplies and/or
services, which do not constitute research, from non-U.S. sources by
U.S. award recipients is permitted.''

(vii) Proposals from non-U.S. PI organizations that propose the
funding of U.S. Co-Is. A proposal submitted by a non-U.S. organization
that involves U.S. Co-Is for whom NASA funding is requested must
provide the budgets for those U.S. Co-Is in compliance with all
applicable provisions in this Section 2.3.10. The budget should
identify the U.S. Co-I organization to which funding will be
awarded. In addition, compliance is required by the proposing
non-U.S. organization with the provisions of Section (l) of Appendix
B.

(viii) Scholarships and student aid costs. If selected, proposers must comply with the policy of the Office of Management and Budget set out in 2 CFR \S200.466, Scholarships and student aid costs. To ensure compliance with this policy, proposers must affirm in their
proposals the
\end{quote}


\subsection{What are the costs?}
- MGS 20\% time in first year, 30\% time in second year.
- JL XX\% time
- Additional postdoc and students at 50-100\% time for N = 2-3 years
- Custom equipment for detector metrology: 50K
- Experimental IR detector FPA and custom ASIC and electronics: 250K ???
- Si boule acquisition, orienting, cutting, lab costs: 50K
- Total: 1.0 - 1.5M ??



%%%%%%%%%%%%%%%%%%%%%%%%%%%%%%%%%%%%%%%%%%%%%%%%%%%%%%%%%%%%%%%%%%%%%%%%%%%%%%%%
%% Special Notifications and/or Certifications

\cleardoublepage

\section*{Special Notifications and/or Certifications}
\addcontentsline{toc}{section}{Special Notifications and/or Certifications}
%%


%%%%%%%%%%%%%%%%%%%%%%%%%%%%%%%%%%%%%%%%%%%%%%%%%%%%%%%%%%%%%%%%%%%%%%%%%%%%%%%%
%% Table of Personnel and Work Effort

\cleardoublepage

\section*{Table of Personnel and Work Effort}
\addcontentsline{toc}{section}{Table of Personnel and Work Effort}
%%
2.3.13 Table of Personnel and Work Effort

Please note that this section does not apply to proposals resulting in
contracts. The Table of Personnel and Work Effort summarizes the work
effort required to perform the proposed investigation, should it be
selected. The table must include the names and/or titles of all
personnel necessary to perform the proposed effort, regardless of
whether they require funding. Where names are not known, include the
position, such as postdoc or technician. For each individual, list the
planned work to be funded by NASA, per period in fractions of a work
year. In addition, for each individual, include planned work not
funded by NASA, if applicable. This planned work not funded by NASA is
not considered cost sharing as defined in 2 CFR \S 200.29. The Table of
Personnel and Work Effort should include only those resources that are
directly applicable to the proposed research effort and should not
include technical information that belongs in the
Scientific/Technical/ Management Section. The detailed budget section
must still include the work effort being paid by NASA.


%%%%%%%%%%%%%%%%%%%%%%%%%%%%%%%%%%%%%%%%%%%%%%%%%%%%%%%%%%%%%%%%%%%%%%%%%%%%%%%%
%% Small Business Subcontracting Plan

\cleardoublepage

\section*{Small Business Subcontracting Plan}
\addcontentsline{toc}{section}{Small Business Subcontracting Plan}
%%
2.3.14 Subcontracting plans

As set out in subparagraph (a)(4) of Appendix B, any proposal from a
large business concern that may result in the award of a contract,
which exceeds \$5,000,000 and has subcontracting possibilities should
include a small business subcontracting plan in accordance with the
clause at FAR 52.219-9, Small Business Subcontracting
Plan. Subcontract plans for contract awards below \$5,000,000 will be
negotiated after selection.


\end{document}
